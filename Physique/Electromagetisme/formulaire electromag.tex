\documentclass[a4paper,12pt]{article}

% Encodage et langue
\usepackage[utf8]{inputenc}
\usepackage[T1]{fontenc}
\usepackage[french]{babel}

% Mathématiques
\usepackage{amsmath, amssymb, amsfonts}
\usepackage{esint} % double intégrale fermée
\usepackage{physics}
\usepackage{bm} % bold maths

% Graphiques et figures
\usepackage{graphicx}
\usepackage{enumitem} % pour les puces de listes

% Mise en page
\usepackage{geometry}
\geometry{a4paper, margin=3cm}
\usepackage{multicol}

% Couleurs et hyperliens
\usepackage{xcolor}
\usepackage{hyperref}
\hypersetup{
    colorlinks=true,
    linkcolor=blue,
    citecolor=blue,
    urlcolor=blue
}

\newcommand{\re}{\textrm}
\newcommand{\Nabla}{\va{\nabla}}
\renewcommand{\curl}{\Nabla \wedge}
\renewcommand{\grad}{\va{\operatorname{grad}}}
\newcommand{\rot}{\va{\operatorname{rot}}}
\renewcommand{\div}{\operatorname{div}}

% Small arrows
\newcommand{\veryshortarrow}[1][4pt]{\mathrel{%
    \hbox{\rule[\dimexpr\fontdimen22\textfont2-.2pt\relax]{#1}{.5pt}}%
    \mkern-4mu\hbox{\usefont{U}{lasy}{m}{n}\symbol{41}}}}
\makeatletter
\setbox0\hbox{$\xdef\scriptratio{\strip@pt\dimexpr
\numexpr(\sf@size*65536)/\f@size sp}$}
\newcommand{\scriptveryshortarrow}[1][4pt]{{%
    \hbox{\rule[\scriptratio\dimexpr\fontdimen22\textfont2-.2pt\relax]
    {\scriptratio\dimexpr#1\relax}{\scriptratio\dimexpr.5pt\relax}}%
    \mkern-4mu\hbox{\let\f@size\sf@size\usefont{U}{lasy}{m}{n}\symbol{41}}}}
\makeatother

\title{Formulaire d'Electromagnétisme}
\author{Raphaël Jamann}
\date{}

\begin{document}
    \maketitle


    \section{Opérateurs de la Théorie des Champs}

        \subsection{Opérateur Nabla $\nabla$}
            L'opérateur nabla est défini en coordonnées cartésiennes par :

            $$ \nabla = \begin{pmatrix}
                    \pdv{}{x} \\[0.5em]  % ajoute de l'espace verticalement
                    \pdv{}{y} \\[0.5em]
                    \pdv{}{z}
                \end{pmatrix} 
            $$

            Il sera utile pour l'écriture d'autres opérateurs tels que $\grad$, $\div$ ou encore $\rot$.

        \subsection{Le gradiant}
            L'opérateur gradient ne s'applique qu'aux champs scalaires $f(x, y, z)$ 
            et donne un champ de vecteurs correspondant à la variation locale du champ scalaire $f$ dans l'espace. 

             On peut le noter de deux manières différentes :

                $$ \grad f = \nabla f(x, y, z) = 
                \begin{pmatrix}
                    \pdv{f}{x} \\[0.5em]  % ajoute de l'espace verticalement
                    \pdv{f}{y} \\[0.5em]
                    \pdv{f}{z}
                \end{pmatrix} $$
                

            Ainsi, le gradient associe à chaque point de l'espace un vecteur orienté 
            dans la direction de la plus forte croissance de $f$ et de norme proportionnelle 
            à la rapidité de cette croissance.

        \clearpage

        \subsection{La divergence}

            L'opérateur divergence s'applique aux champs de vecteurs 
            $ \va{A}(x, y, z) $ et donne un champ scalaire correspondant au taux de « source » ou de « puits » du champ en un point.

            On peut l'écrire de deux manières différentes :

            \[
            \div{\va{A}} = \Nabla \vdot \va{A} 
            = \pdv{A_x}{x} + \pdv{A_y}{y} + \pdv{A_z}{z}
            \]

            Ainsi, la divergence mesure la tendance d'un champ vectoriel à « sortir » (divergence positive) 
            ou à « entrer » (divergence négative) d'un volume infinitésimal autour du point considéré.
            
            La divergence en un point donné $M$ peut donc être interprétée comme le flux par unité
            de volume passant à travers une boîte élementaire délimitant un volume infinitésimal $ \dd \tau $
            tendant vers 0, construit autour de ce point $M$.


        \subsection{Le rotationnel}

            L'opérateur rotationnel (ou \emph{curl}) s'applique également aux champs de vecteurs 
            $ \va{A}(x, y, z) $ et donne un champ de vecteurs représentant la tendance locale 
            du champ à « tourner » autour d'un point.

            On peut l'écrire de deux manières différentes :

            $$ \rot \va{A} = \curl \va{A} =
                \begin{pmatrix}
                    \pdv{A_z}{y} - \pdv{A_y}{z} \\[0.5em]
                    \pdv{A_x}{z} - \pdv{A_z}{x} \\[0.5em]
                    \pdv{A_y}{x} - \pdv{A_x}{y}
                \end{pmatrix} $$

            Ainsi, le rotationnel mesure la « rotation locale » du champ autour d'un point donné.

        \subsection{Flux d'un champ de vecteurs à travers une surface}
            Flux d'un champ de vecteurs à travers une surface élémentaire $ \va{\dd S} $ : 
                $$ \dd \varphi = \va{A} \vdot \va{\dd S} $$
            
            Flux total à travers n'importe quelle surface $S$, plane ou non :
                $$ \varphi = \iint_S \va{A} \vdot \va{\dd S} $$ 

            Un champ de vecteurs dont la divergence est nulle en tout point est appelé
            champ de vecteurs à \textbf{flux conservatif} : $ \div{\va{A}} = 0 $
        
        \subsection{Théorème d'Ostrogradsky}
            
            $$ \varphi = \oiint_S \va{A} \vdot \va{\dd S} = \iiint_\tau \div{\va{A}} \, \dd \tau$$
        
        \subsection{Circulation d'un champ de vecteurs}
            
            La circulation d'un champ de vecteurs $ \va{A} $ le long d'un contour $ \Gamma $ (fermé ou non) se définit par :
                $$ \mathcal{C} = \int_\Gamma \va{A} \vdot \va{\dd l} $$
            Lorsque le contour est fermé, on note :
                $$ \mathcal{C} = \oint_\Gamma \va{A} \vdot \va{\dd l} $$
            La circulation est dite \textbf{consevative} si elle est nul sur un contour fermé $ \Gamma $ :
                $$ \mathcal{C} = \oint_\Gamma \va{A} \vdot \va{\dd l} = 0 $$

            
                

        \subsection{Théorème de Stokes}
            
            $$ \oint_{\Gamma^+} \va{A} \vdot \va{\dd l} = \iint_S \rot{\va{A}} \vdot \va{\dd S} $$

\end{document}