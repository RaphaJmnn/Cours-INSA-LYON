\documentclass{article}

% Language setting
% Replace `english' with e.g. `spanish' to change the document language
\usepackage[french]{babel}
\usepackage[T1]{fontenc}

% Set page size and margins
% Replace `letterpaper' with`a4paper' for UK/EU standard size
\usepackage[letterpaper,top=0.5cm,bottom=0.5cm,left=2cm,right=2cm,marginparwidth=1.75cm]{geometry}

% Useful packages
\usepackage{amsmath}
\usepackage{graphicx}
\usepackage[colorlinks=true, allcolors=blue]{hyperref}
\usepackage{wrapfig}

\usepackage{tikz}
\usepackage[
  european resistor,
  RPvoltages,
  european current,
  european voltage,
  straightvoltages
]{circuitikz}
\usetikzlibrary{babel}

% Colonnes
\newenvironment{col}[1]
{\begin{minipage}[t]{\dimexpr \textwidth * #1/100 - 0.03\textwidth}}{\end{minipage}\hspace{0.03\textwidth}}
\newenvironment{colf}[1]
{\begin{minipage}[t]{\dimexpr \textwidth * #1/100}}{\end{minipage}}
\newcommand{\twoCol}[3][50]{
    \begin{col}{#1}
        #2
    \end{col}
    \begin{colf}{\numexpr 100 - #1\relax}
        #3
    \end{colf}
}

\title{TP Mesure en Régime Transitoire}
\author{Raphaël Jamann}
\date{} % remove the date

\begin{document}
    \maketitle

    \section{Objectif}

    \thispagestyle{empty} % remove page numbering
    \pagestyle{empty}

    \textbf{A l'aide du matériel à votre disposition, vous devez proposer une méthode permettant de
    déterminer la valeur de la capacité et de la résistance inconnues en exploitant le régime transitoire
    (i.e. décharge du condensateur).}

    \section{Montage expérimental}
    
    \begin{wrapfigure}[7]{r}{0.45\textwidth}
        \centering
        \vspace{-2.3cm}
        \begin{circuitikz}
            % Square wave voltage source (TTL: 0-5V)
            \draw (0,0)
                to[square voltage source, v=$0/5V$] (0,3) -- (1.3,3) 
                to[R=$10k\Omega$, l=$U_R$] (2.7,3) -- (4,3) -- (4,2.2)  % 10kΩ resistor with voltage label
                to[R=$5k\Omega$, l=$U_{R_X}$] (4,0.8) -- (4,0) -- (3,0)   % 5kΩ resistor with voltage label
                to[C=$100nF$, l=$C$] (1,0) -- (0,0); % 100nF capacitor with voltage label
            
            % Oscilloscope box
            \draw (1.2,-1.5) rectangle (2.8,-1);
            \node at (2,-1.25) {\tiny\textbf{Oscilloscope}};
            \node[right, red] at (3.2,-0.7) {\tiny CH1};
            \node[left] at (0.8,-0.7) {\tiny Ground};

            % Connection to circuit
            \draw[red] (3.2,-1.25) -- (3.2,0); 
            \draw[red] (3.2,-1.25) -- (2.8,-1.25);
            \draw (0.8,-1.25) -- (0.8,0); 
            \draw (0.8,-1.25) -- (1.2,-1.25);
            \fill[red] (3.2,0) circle (2pt); % Connection dot
            \fill (0.8,0) circle (2pt); % Ground dot

        \end{circuitikz}
    \end{wrapfigure}

    On étudie les deux montages ci-dessous.
    Le générateur basse fréquence est branché au circuit par la sortie
    \textit{TTL}\footnote{\textbf{TTL:} Transistor Transistor Logic}. Il délivre alors un signal
    créneau passant de 0$\,V$ à 5$\,V$ à une fréquence de 50$\,Hz$ (période de 20$\, ms$).
    On étudie la décharge du condensateur, à $t=0\,s$ le GBF passe de 5$\,V$ à 0$\,V$.
    C'est pourquoi il n'interviendra pas dans les calculs suivant (loi des mailles).
   
    \vspace{1cm}
    \hrule
    \vspace{3mm}
    \twoCol{        
        \begin{center}
            \begin{circuitikz}
                % Square wave voltage source (TTL: 0-5V)
                \draw (0,0)
                    to[square voltage source] (0,3) -- (1.3,3) 
                    to[R=$10k\Omega$, v=$U_R$] (2.7,3) -- (4,3) -- (4,2.2)  % 10kΩ resistor with voltage label
                    to[R=$5k\Omega$, v=$U_{R_X}$] (4,0.8) -- (4,0) -- (3,0)   % 5kΩ resistor with voltage label
                    to[C=$100nF$, v>=$U_C$] (1,0) -- (0,0); % 100nF capacitor with voltage label
            \end{circuitikz}

            \noindent\textbf{Loi des mailles:} 
            \begin{align*}
                &\Longleftrightarrow &\quad -U_R - U_{R_X} + U_C &= 0 \\
                &\Longleftrightarrow &\quad -(R+R_X)I + U_C &= 0 \quad \text{avec } I=-C_X\frac{\mathrm{d}U_C}{\mathrm{d}t} \\
                &\Longleftrightarrow &\quad C_X(R+R_X)\frac{\mathrm{d}U_C}{\mathrm{d}t} + U_C &= 0 \\
                &\Longrightarrow &\quad \tau_1=C_X(R+R_X)
            \end{align*}
        \end{center}
    }{
        \begin{center}
            \begin{circuitikz}
                % Square wave voltage source (TTL: 0-5V)
                \draw (0,0)
                    to[square voltage source] (0,3) -- (4,3) -- (4,2.2)
                    to[R=$5k\Omega$, v=$U_{R_X}$] (4,0.8) -- (4,0) -- (3,0)   % 5kΩ resistor with voltage label
                    to[C=$100nF$, v>=$U_C$] (1,0) -- (0,0); % 100nF capacitor with voltage label
            \end{circuitikz}

            \noindent\textbf{Loi des mailles:} 
            \begin{align*}
                &\Longleftrightarrow &\quad -U_{R_X} + U_C &= 0 \\
                &\Longleftrightarrow &\quad -R_X I + U_C &= 0 \\
                &\Longleftrightarrow &\quad C_X R_X \frac{\mathrm{d}U_C}{\mathrm{d}t} + U_C &= 0 \\
                &\Longrightarrow &\quad \tau_2=R_X C_X
            \end{align*}
        \end{center}
    }

    \vspace{5mm}
    \begin{center}
        \textit{On peut alors faire la mesure de $\tau_1$ puis $\tau_2$ et on résout le système pour obtenir $C_X$ et $R_X$.}\\
        \href{https://falstad.com/circuit/circuitjs.html?ctz=CQAgTADCEHQf8CsICcMDsAOAzClmAWRbANkzAEZsREpkDrEBTAWgooCgA3EbMETFHb9B0cDShgYyKciixEHAE4hhqlPz7r+dBBwDGvfhQ1GBUAqrgIbt+BVbpo1iBUTp3EDwQLkweEmd7aANeUm0wwNFLChc7OwcWJ1g3dHRsRAIUDzwIMEQCoNcQlQIKKItyiKEbbkjzEDKKsX5acGl2uWdFUqqTTXD+sVp4DgB3VQpjUzVRCA4AD3EwQIoicBRG8EsAEQBLgBcmJQBbAEsAO2OAHQBnQFYNwHh9wECCRd54bcaGcCdLMF3DsdzlclHcnoAggjeJAo4DA1EwgTAKy2-xAACVAMkEb1IiLhvBQMKRgT+lkxbwY1DWxLaa02JPRHAA9i0xAQIPgaEF8qyxNQdOJsEzOfy2Ry5DAYTDLMhqNQoLKOEA}
        {Lien pour réaliser le tp avec une simulation.}
    \end{center}

    

    \[
        \left\{    
            \begin{aligned}
                R_X &= \dfrac{\tau_2}{C_X} \\
                \tau_1 &= C_X\left(R+\dfrac{\tau_2}{C_X}\right)
            \end{aligned}
        \right. 
        \quad \Longleftrightarrow\quad \left\{    
            \begin{aligned}
                R_X &= \dfrac{R\,\tau_2}{\tau_1-\tau_2} \\
                C_X &= \dfrac{\tau_1-\tau_2}{R}
            \end{aligned}
        \right. 
    \]

    \vspace{1cm}

    \noindent\textbf{Réponses:}
    Les valeurs mesurées de $\tau_1$ et $\tau_2$ sont respectivement $0.5\,ms$ et $1.5\,ms$.\\
    On obtient alors $C_X=100\,nF$ et $R_X=5\,k\Omega$

\end{document}