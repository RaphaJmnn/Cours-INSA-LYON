\documentclass{article}

% Language setting
% Replace `english' with e.g. `spanish' to change the document language
\usepackage[french]{babel}
\usepackage[T1]{fontenc}

% Set page size and margins
% Replace `letterpaper' with`a4paper' for UK/EU standard size
\usepackage[letterpaper,top=0.5cm,bottom=0.5cm,left=2cm,right=2cm,marginparwidth=1.75cm]{geometry}

% Useful packages
\usepackage{amsmath}
\usepackage{graphicx}
\usepackage[colorlinks=true, allcolors=blue]{hyperref}
\usepackage{wrapfig}

\usepackage{tikz}
\usepackage[european, straightvoltages, RPvoltages, cute inductor]{circuitikz}  % RPvoltages definie la convention des dipoles (sens tension faux sinon)
\usetikzlibrary{babel}

% Colonnes
\newenvironment{col}[1]
{\begin{minipage}[t]{\dimexpr \textwidth * #1/100 - 0.03\textwidth}}{\end{minipage}\hspace{0.03\textwidth}}
\newenvironment{colf}[1]
{\begin{minipage}[t]{\dimexpr \textwidth * #1/100}}{\end{minipage}}
\newcommand{\twoCol}[3][50]{
    \begin{col}{#1}
        #2
    \end{col}
    \begin{colf}{\numexpr 100 - #1\relax}
        #3
    \end{colf}
}

\title{Filtres Passifs}
\author{Raphaël Jamann}
\date{} % remove the date

\begin{document}
    \maketitle

    Lien playliste Youtube pour comprendre les filtres: \href{https://youtube.com/playlist?list=PLBRwdavN8jtvGuPq0_vqaGfYz1iMiCGmt&si=m3tA0icXFmadQLan}{Playliste}.
    Aide Latex circuit: \href{https://nboulaire.developpez.com/tutoriels/latex/circuitikz_base/}{lien}.

    \section{Qu'est ce qu'un filtre}
    
    Un filtre est un quadripôle linéaire (constitué de dipôles linéaires R,L et C) qui \textbf{permet
    d'atténuer certaines fréquences} en régime sinusoïdal. 

    \bigskip
    Les filtres fonctionnent grâce à l'impédance complexe des dipôles R et C qui dépendent
    de la pulsation $\omega$ et donc de la fréquence $f=\frac{\omega}{2\pi}$. \\
    En effet, l'impédance d'un condensateur est $\underline{Z}_C=\dfrac{1}{j\omega C}$ . \\
    L'impédance d'une bobine L est $\underline{Z}_L=j\omega L$


    \section{Exemples de filtres passe haut}

    \twoCol{      

        \subsection{High Pass RC Filter}

        \centering
        \begin{circuitikz}        
            % Circuit code
            \draw (0,0) to[short,o-o] ++ (4,0);
            \draw (0,2) to[C, l=C, o-] ++ (3,0) coordinate(a);
            \draw (a) to[short, -o] ++ (1,0);
            \draw (a) to[R, name=R, *-*] ++(0,-2);
            \node at (R.center) {R};  % draw label "R" at the center of the resistance

            % Voltage labels
            \draw (0,2) to[open,v=V$_{\text{in}}$\;] ++(0,-2);
            \draw (4,2) to[open,v^=\hspace{1.5mm} V$_{\text{out}}$] ++(0,-2);
        \end{circuitikz}

        Sur ce montage, lorsque la fréquence est basse, l'impédance du condensateur est très
        grande donc la tension de sortie est plus faible que celle d'entrée.
    }{
        \subsection{High Pass RL Filter}

        \centering
        \begin{circuitikz}        
            % Circuit code
            \draw (0,0) to[short,o-o] ++ (4,0);  % draw the bottom wire
            \draw (0,2) to[R, name=R, o-] ++ (3,0) coordinate(a);  % draw the resistance
            \node at (R.center) {R};  % draw label "R" at the center of the resistance
            \draw (a) to[short,-o] ++ (1,0);  % draw wire to the right of R
            \draw (a) to[L, l_=L, *-*] ++(0,-2);  % draw the Capacitor

            % Voltage labels
            \draw (0,2) to[open,v_=V$_{\text{in}}$\;] ++(0,-2);
            \draw (4,2) to[open,v^=\hspace{1.5mm} V$_{\text{out}}$] ++(0,-2);
        \end{circuitikz}
    }



    \section{Exemples de filtres passe bas}

    \twoCol{      

        \subsection{Low Pass RC Filter}

        \centering
        \begin{circuitikz}        
            % Circuit code
            \draw (0,0) to[short,o-o] ++ (4,0);  % draw the bottom wire
            \draw (0,2) to[R, name=R, o-] ++ (3,0) coordinate(a);  % draw the resistance
            \node at (R.center) {R};  % draw label "R" at the center of the resistance
            \draw (a) to[short,-o] ++ (1,0);  % draw wire to the right of R
            \draw (a) to[C, l_=C, *-*] ++(0,-2);  % draw the Capacitor

            % Voltage labels
            \draw (0,2) to[open,v_=V$_{\text{in}}$\;] ++(0,-2);
            \draw (4,2) to[open,v^=\hspace{1.5mm} V$_{\text{out}}$] ++(0,-2);
        \end{circuitikz}

        On étudie le montage ci-contre. \\
        lien: \href{https://www.falstad.com/circuit/circuitjs.html?ctz=CQAgjCAMB0l3BWK0AckDMYwE4As3sA2SQgdgCYFsQFIaa6EBTAWiwCgA3EXdckSoR58eYIXQhh49OrOgJ2AJ2H9BNUkLV1ykdgHd1Q3GJWjx7AMaGBCIQg1meyeJAjloU9IQT40hNNhYpM6uUPqmaugouDbmBrz8xkJRMUlhBvaatiCE5DFa4ZmOuakmugAeIOiQSDhCpFjgBE4mAIIAOgDOYADWABIAXl0A9gB2XQAOTACuAC5dnMMAlopdAI7TTF0ANgCGXbNMo51LY10AJludncOKs0tXTJ3znbuzswCXo9MfTNDslWMKHAPhoKGCYHsTjyIAAYkttrNFFcJrtrlcAEZogECNBVXhVUh0dC4IwCGIANS6RyRvxxhFwSBJMQoyVwwXylK6NzuD3Ywyg4EFuEg2GBSBgkEogv46DCQA}{lien}.
    
    }{
        \subsection{Low Pass RL Filter}

        \centering
        \begin{circuitikz}        
            % Circuit code
            \draw (0,0) to[short,o-o] ++ (4,0);
            \draw (0,2) to[L, l=L, o-] ++ (3,0) coordinate(a);
            \draw (a) to[short, -o] ++ (1,0);
            \draw (a) to[R, name=R, *-*] ++(0,-2);
            \node at (R.center) {R};  % draw label "R" at the center of the resistance
    
            % Voltage labels
            \draw (0,2) to[open,v=V$_{\text{in}}$\;] ++(0,-2);
            \draw (4,2) to[open,v^=\hspace{1.5mm} V$_{\text{out}}$] ++(0,-2);
        \end{circuitikz}

        Lien pour tester ce filtre: \href{https://www.falstad.com/circuit/circuitjs.html?ctz=CQAgjCAMB0l3BWc0BsBmSAOMB2HAWBATk0xXIgUhCSpoFMBaMMAKADcQUAmfEbhCi68QaTH2oQw8GlDkwErAO7C+YvjgTdR4qKwA2q-oJCbtAoZKixEkMGnyYsaFNkiuJrAE6mtxoWY6EuDwrAAepkQ4otxCmDiYMUQgfCIAagA6AM5ZAPZeAC4AlvThKZoxQoJIaNzJqXyZWfQAdgVeAJelKoEWIE7UfZDKvtrq-XBBeio8qSYIOEJDI7NTC0LjwxHEyfbJi4lg+NoNIABiRfrt9NkADgCGOTdZAEaPZY6HRNSYRHxgRDQKXAQgAgtkwABrAASAC9srkWnd6ABXArZdi5IpebIARxRz3092yBVaWSKiOyABNnnlCiVsvQsuisvcCgUOi0UV1oKxcnIINR8JASLIYILgdQgUCpaJWEA}{simulation}.

    }

\end{document}