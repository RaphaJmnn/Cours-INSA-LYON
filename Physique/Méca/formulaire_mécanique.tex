\documentclass[a4paper,12pt]{article}

% Encodage et langue
\usepackage[utf8]{inputenc}
\usepackage[T1]{fontenc}
\usepackage[french]{babel}

% Mathématiques
\usepackage{amsmath, amssymb, amsfonts}
\usepackage{siunitx} % Pour les unités
\sisetup{locale=FR} % Configuration pour le français
\usepackage{bm} % bold maths

% Graphiques et figures
\usepackage{graphicx}
\usepackage{enumitem} % pour les puces de listes

% Mise en page
\usepackage{geometry}
\geometry{a4paper, margin=1.5cm}
\usepackage{multicol}

% Couleurs et hyperliens
\usepackage{xcolor}
\usepackage{hyperref}
\hypersetup{
    colorlinks=true,
    linkcolor=blue,
    citecolor=blue,
    urlcolor=blue
}

\newcommand{\re}{\textrm}
\newcommand{\diff}{\mathrm{d}} % Pour les différentielles droites
\renewcommand{\vec}{\overrightarrow}  % change vector arrows

% Small arrows
\newcommand{\veryshortarrow}[1][4pt]{\mathrel{%
    \hbox{\rule[\dimexpr\fontdimen22\textfont2-.2pt\relax]{#1}{.5pt}}%
    \mkern-4mu\hbox{\usefont{U}{lasy}{m}{n}\symbol{41}}}}
\makeatletter
\setbox0\hbox{$\xdef\scriptratio{\strip@pt\dimexpr
\numexpr(\sf@size*65536)/\f@size sp}$}
\newcommand{\scriptveryshortarrow}[1][4pt]{{%
    \hbox{\rule[\scriptratio\dimexpr\fontdimen22\textfont2-.2pt\relax]
    {\scriptratio\dimexpr#1\relax}{\scriptratio\dimexpr.5pt\relax}}%
    \mkern-4mu\hbox{\let\f@size\sf@size\usefont{U}{lasy}{m}{n}\symbol{41}}}}
\makeatother

\title{Formulaire de Mécanique}
\author{Raphaël Jamann}
\date{}

\begin{document}
    \maketitle

    \section{Les différentes forces}

    \begin{multicols}{2}
        
        \subsection{Force de Pesanteur (Poids)}
            $$ F_p = m g $$
            \begin{itemize}[label=$\bullet$]
                \item $F_p$ : force de pesanteur.
                \item $m$ : masse de l'objet (en kilogrammes).
                \item $g$ : accélération due à la gravité ($9.81 \, m \, s^{-2}$).
            \end{itemize}
            
        \subsection{Force de Réaction Normale}
                $$ N = F_p \cos(\theta) $$
            \begin{itemize}[label=$\bullet$]
                \item $N$ : force normale.
                \item $F_p$ : force de pesanteur.
                \item $\theta$ : angle entre la surface et l'horizontale.
            \end{itemize}
            
        \subsection{Force de Frottement Solide}
                $$ F_f = \mu N $$
            \begin{itemize}[label=$\bullet$]
                \item $F_f$ : force de frottement.
                \item $\mu$ : coefficient de frottement (statique ou cinétique).\\
                $\mu_c=\frac{R_T}{R_N}$ quand le système glisse et $\mu_s=\frac{R_{T_{max}}}{R_N}$ quand il est statique.
                \item $N$ : force normale.
            \end{itemize}
            
        \subsection{Force Élastique (Loi de Hooke)}
                $$ \vec{F} = -k (L- l) \vec{u} $$
            \begin{itemize}[label=$\bullet$]
                \item $F$ : force de rappel d'un ressort.
                \item $k$ : constante de raideur (en $N , m^{-1}$).
                \item $l$ : longueur "à vide" du ressort.
                \item $L$ : longueur du ressort étiré ou compressé.
                \item $\vec{u}$ : vecteur dirigé de l'extrémité fixe du ressort vers le système (l'autre extrémité).
            \end{itemize}
            
        \subsection{Force Gravitationnelle}
                $$ F_g = G \frac{m_1 m_2}{r^2} $$
            \begin{itemize}[label=$\bullet$]
                \item $F_g$ : force gravitationnelle.
                \item $G$ : constante de gravitation universelle ($6,67 \times 10^{-11} \ N \ (m^2 \, kg^{-2})$).
                \item $m_1$ et $m_2$ : masses des objets.
                \item $r$ : distance entre les centres de masse.
            \end{itemize}

        \subsection{Force de Lorentz}
            \hspace{5.3cm} \textit{magnétique}
            $$ \vec{F} = q(\vec{E} + \vec{v} \wedge \vec{B}) = \underbrace{q\vec{E}} + \overbrace{q \vec{v} \wedge \vec{B}} $$
            \hspace{4.5cm} \textit{électrique}

            Une charge $q$ qui se déplace dans une région de l'espace où coexistent 
            un champ électrique $\vec{E}$ et un champ magnétique $\vec{B}$, subit une force 
            totale dite de Lorentz.

        \subsection{Force de Laplace}
            Soit un fil rectiligne de longueur $l$ parcouru par un 
            courant $I$. Si ce fil est situé dans une région de 
            l'espace où il y a un champ magnétique $\vec{B}$, alors il 
            est soumis à une force de Laplace donnée par :
            $$ \vec{F} = I \cdot \vec{l} \wedge \vec{B} $$
    
    \end{multicols}

    \section{Les Lois de Newton}

        \begin{itemize}[label=$\bullet$]
            \item \textbf{1ère loi de Newton : principe d'inertie}\\
            Si $R_g$ est galiléen, alors tout référentiel en translation
            rectiligne uniforme par rapport à $R_g$ est aussi galiléen.
                $$ \frac{\diff \varepsilon_c}{\diff t} = 0 \quad\quad \frac{\diff \vec{p}}{\diff t} = \vec{0} \quad\quad \frac{\diff \vec{\sigma}}{\diff t} = \vec{0} $$

            \item \textbf{2ème loi de Newton : principe fondamental de la dynamique (PFD)} \\
            Les forces sont à l'origine du mouvement : $\sum \vec{F_{ext}} = \dfrac{\diff \vec p}{\diff t} = m \vec{a}$ dans un référentiel galiléen.

            \item \textbf{3ème loi de Newton : principe des actions réciproques} \\
            Si deux points sont en interaction, on a deux forces : $\vec{f_{1/2}} = - \vec{f_{2/1}}$\\
            Une force unique existant seule n'existe pas, d'où la résultante nulle des forces intérieures à un système.
        \end{itemize}


    \section{Définitions}

    \begin{multicols}{2}
        
        \subsection{Quantité de mouvement}
            La quantité de mouvement (ou résultante cinétique) est définie comme :
                $$\vec{p} = m \vec{v}$$

        \subsection{Moment cinétique}
            Le moment cinétique est le moment de la quantité de mouvement $\vec{p}$.
                $$ \vec \sigma = \vec{OM} \wedge \vec p = \vec{OM} \wedge m \vec v $$
            Ou alors on la définie comme ceci par rapport à un axe $\Delta$ :
                $$ \sigma_\Delta = J_\Delta \omega = J_\Delta \dot \theta $$ 
        
        \subsection{Moment d'inertie}
            Le moment d'inertie d'un solide (domaine $\mathcal{D}$) par rapport à un axe fixe $\Delta$ est:
                $$ J_\Delta = \iiint_{\mathcal{D}} r^2 dm $$



        \subsection{Énergie cinétique}
            L'énergie cinétique d'un point matériel de masse $m$ et de vitesse $v$ est : 
                $$ E_c = \frac{1}{2} m v^2 $$

            Cas du mouvement de rotation autour d'un axe fixe $\Delta$ :
                $$ E_c = \frac{1}{2} J_\Delta \omega^2 $$

        \subsection{Énergie Potentiel}

            Si le travail élémentaire de la force $F$ peut s'écrire sous la
            forme d'une différentielle exacte alors, on peut définir
            une énergie potentielle $E_p$ telle que :
            $$ \diff E_p = -\diff W $$
            On dit que la force dérive d'une énergie potentielle et 
            aussi que cette force est conservative.

        \subsection{Énergie mécanique}

            Définition de l'énergie mécanique :
            $$E_M = E_C + \sum E_P$$
        

        \subsection{Travail d'une force}
            Le travail mécanique représente la quantité d'énergie échangée entre le
            système et le milieu extérieur.
            $$ \delta W = \vec F \cdot \vec{\diff l} \quad \quad W_{A\scriptveryshortarrow B}(\vec F) = \int_A^B \vec{F} \cdot \vec{\diff l}$$

            Dans le cas d'une rotation élémentaire du point $M$ autour de $(Oz)$:
            $$ \delta W = \mathcal{M}_{Oz}(\vec{F}) \diff \theta $$

            \begin{itemize}[label=$\bullet$]

                \item \textbf{Travail gravitationnel :} $ \delta W_{\vec P} = m g h $

                \item \textbf{Travail élastique :}\\
                    Ressort : $ \delta W_{\vec P_e} = \frac{1}{2} k x^2 $ avec $x$ l'allongement (ou raccourcicement) du ressort.\\
                    Torsion : $ \delta W_{\vec P_e} = \frac{1}{2} C \theta^2 $
            \end{itemize}


        \subsection{Puissance d'une force}

            La puissance d'une force F, qui s'exerce sur un
            point $M$ de vitesse $\vec{v}$ est définie par :

            $$ P = \vec{F}\cdot \vec{v} $$

            \textbf{Lorsque la puissance d'une force est nulle ($P=0$), la force peut dévier
            la trajectoire du système mais pas modifier la norme de sa vitesse.}


    \end{multicols}

    \section{Les théorèmes importants}
        
        \subsection{Théorème de l'énergie mécanique}

        Dans un référentiel galiléen, la variation d'énergie mécanique d'un point matériel est égale à la somme des travaux des forces non-conservatives :

        $$ \Delta E_M = E_M(B) - E_M(A) = \sum_i W_{A \scriptveryshortarrow B}(\vec{F}_{i,\textbf{nc}}) = W_{A \scriptveryshortarrow B}(\vec{F}_{ext,\textbf{nc}}) $$


        \subsection{Théorème de l'énergie cinétique (TEC)}

            Dans un référentiel galiléen, la variation de l'énergie cinétique d'un point matériel est égale à la somme des travaux de toutes les forces :

                $$ \Delta E_C = E_C(B) - E_C(A) = \sum_i W_{A \scriptveryshortarrow B}(\vec{F_i}) = W_{A \scriptveryshortarrow B}(\vec{F_{ext}}) $$

                $$ \diff E_c = \delta W_{(\vec{F_{tot}})} $$


        \subsection{Théorème du moment cinétique (TMC)}
            Pour un solide indéformable en rotation autour d'un axe fixe $\Delta$ et soumis à plusieurs moments de forces
            extérieures $ \mathcal{M}_{\Delta} (\vec F) $ par rapport à $\Delta$, on a :
                $$\sum \mathcal{M}_{\Delta} (\vec F) = J_\Delta \ddot \theta = J_\Delta \dot \omega $$

            Ou alors dans un référentiel galiléen, la dérivée par rapport au temps du 
            moment cinétique $\frac{\diff \vec{\sigma_O}}{\diff t}$ du point matériel calculé en $O$ (point fixe) est égal 
            au moment résultant $\vec{\Gamma_O}$ par rapport à $O$.

                $$ \frac{\diff \vec{\sigma_O}}{dt} = \sum \vec{\mathcal{M}_{0 (\vec{F}_{ext})}} $$

            Attention cependant $\vec{\Gamma_O} \neq \vec{OM} \wedge \vec{R}$ !

        

        \subsection{Théorème de la puissance cinétique}

            Dans un référentiel galiléen, la dérivée par rapport au
            temps de l'énergie cinétique d'un point matériel est égale
            à la puissance de toutes les forces appliquées.
             $$ P_{tot} = \frac{\diff E_c}{\diff t} $$
\end{document}