\documentclass[a4paper,12pt]{article}

% Encodage et langue
\usepackage[utf8]{inputenc}
\usepackage[T1]{fontenc}
\usepackage[french]{babel}

% Mathématiques
\usepackage{amsmath, amssymb, amsfonts}
\usepackage{siunitx} % Pour les unités
\sisetup{locale=FR} % Configuration pour le français
\usepackage{bm} % bold maths

% Graphiques et figures
\usepackage{graphicx}
\usepackage{enumitem} % pour les puces de listes

% Mise en page
\usepackage{geometry}
\geometry{a4paper, margin=1.5cm}
\usepackage{multicol}

% Couleurs et hyperliens
\usepackage{xcolor}
\usepackage{hyperref}
\hypersetup{
    colorlinks=true,
    linkcolor=blue,
    citecolor=blue,
    urlcolor=blue
}

\newcommand{\re}{\textrm}
\newcommand{\diff}{\mathrm{d}} % Pour les différentielles droites

% Small arrows
\newcommand{\veryshortarrow}[1][4pt]{\mathrel{%
    \hbox{\rule[\dimexpr\fontdimen22\textfont2-.2pt\relax]{#1}{.5pt}}%
    \mkern-4mu\hbox{\usefont{U}{lasy}{m}{n}\symbol{41}}}}
\makeatletter
\setbox0\hbox{$\xdef\scriptratio{\strip@pt\dimexpr
\numexpr(\sf@size*65536)/\f@size sp}$}
\newcommand{\scriptveryshortarrow}[1][4pt]{{%
    \hbox{\rule[\scriptratio\dimexpr\fontdimen22\textfont2-.2pt\relax]
    {\scriptratio\dimexpr#1\relax}{\scriptratio\dimexpr.5pt\relax}}%
    \mkern-4mu\hbox{\let\f@size\sf@size\usefont{U}{lasy}{m}{n}\symbol{41}}}}
\makeatother

\title{Formulaire de Thermodynamique}
\author{Raphaël Jamann}
\date{}

\begin{document}
    \maketitle

    \section{Les Lois de Newton}

        \begin{itemize}
            \item \underline{\textbf{2ème loi de Newton : principe fondamental de la dynamique (PFD)}}\\
            Les forces sont à l'origine du mouvement : $\sum \vec F_\re{ext} = \dfrac{\re d \vec p}{\re d t} = m \vec a$ dans un référentiel galiléen.\\

            \item \underline{\textbf{3ème loi de Newton : principe des actions réciproques}}\\
            Si deux points sont en interaction, on a deux forces : $\vec{f_\re{1/2}} = - \vec{f_\re{2/1}}$\\
            Une force unique existant seule n'existe pas, d'où la résultante nulle des forces intérieures à un système.
        \end{itemize}


    \section{Définitions}

        \subsection{Moment cinétique}

            \begin{flalign*}
                \vec \sigma &= \vec{OM} \wedge \vec p&\\
                &= \vec{OM} \wedge m \vec v&\\
                &= r \vec{e_r} \wedge m r \dot\theta \vec{e_\theta}&\\
                &= m r^2 \dot\theta \vec{e_z}&\\
                &= J \vec \Omega&
            \end{flalign*}

            \begin{itemize}
                \item Moment d'inertie :\\ $J = m r^2$ \quad (grandeur d'inertie)\vspace{4pt}
                \item Vecteur rotation instantané :\\ $\vec \Omega = \dot\theta \vec{e_z}$ \quad (grandeur cinématique)
            \end{itemize}

            \[\dfrac{\re d \vec \sigma}{\re d t} = \sum \vec{\mathcal{M}_{ext}}\]


        \subsection{Énergie cinétique}

            \[\varepsilon_c = \dfrac{1}{2} m v^2 \hspace*{4em} \dfrac{\re d \varepsilon_c}{\re d t} = \sum P_{ext}\]

        
        \subsection{Énergie mécanique}

            Définition de l'énergie mécanique :
            \[E_M = E_C + \sum E_P\]

        \subsection{Quantité de mouvement}

            \begin{align*}
                \vec p = m \vec v \hspace*{4em} \dfrac{\re d \vec p}{\re d t} &= \sum \vec{F_{ext}}\\
                m \vec a &= \sum \vec{F_{ext}}
            \end{align*}
        

        \subsection{Travail d'une force}

            Le travail mécanique représente la quantité d'énergie échangée entre le système et le milieu extérieur et pouvant être transformée d'une forme en une autre.\\

            Pour une force constante, $W_{A\scriptveryshortarrow B}(\vec F) = \vec F \cdot \vec{AB}$\\

            On note aussi, $\delta W = \vec F \cdot \vec{\re d l}$, ce qui donne pour une force quelconque :\\
            \[ W_{A\scriptveryshortarrow B}(\vec F) = \int_A^B \vec F \cdot \vec{\re dl} \]


    \section{Les théorèmes importants}
        
        \subsection{Théorème de l'énergie mécanique}

        Dans un référentiel galiléen, la variation d'énergie mécanique d'un point matériel est égale à la somme des travaux des forces non-conservatives :

        \[\Delta E_M = E_M(B) - E_M(A) = \sum_i W_{A \scriptveryshortarrow B}(\vec{F}_{i,\textbf{nc}}) = W_{A \scriptveryshortarrow B}(\vec{F}_{ext,\textbf{nc}})\]


        \subsection{Théorème de l'énergie cinétique (TEC)}

            Dans un référentiel galiléen, la variation de l'énergie cinétique d'un point matériel est égale à la somme des travaux de toutes les forces :

            \[\Delta E_C = E_C(B) - E_C(A) = \sum_i W_{A \scriptveryshortarrow B}(\vec{F_i}) = W_{A \scriptveryshortarrow B}(\vec{F}_{ext})\]

            \[ \diff E_c = \delta W_{(\vec F)} \]


        \subsection{Théorème du moment cinétique (TMC)}

            Pour un point matériel en rotation autour d'un axe fixe $\Delta$ ($J_\Delta \ddot \theta$ la dérivée du moment cinétique sur $\Delta$) :
            \[\sum \mathcal{M}_{\vec F} (\Delta) = J_\Delta \ddot \theta = J_\Delta \dot \omega \]

\end{document}