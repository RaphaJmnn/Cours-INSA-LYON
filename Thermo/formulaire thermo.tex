\documentclass[a4paper,12pt]{article}

% Encodage et langue
\usepackage[utf8]{inputenc}
\usepackage[T1]{fontenc}
\usepackage[french]{babel}

% Mathématiques
\usepackage{amsmath, amssymb, amsfonts}
\usepackage{siunitx} % Pour les unités
\sisetup{locale=FR} % Configuration pour le français
\usepackage{bm} % bold maths

% Graphiques et figures
\usepackage{graphicx}
\usepackage{enumitem} % pour les puces de listes

% Mise en page
\usepackage{geometry}
\geometry{a4paper, margin=1.5cm}
\usepackage{multicol}

% Couleurs et hyperliens
\usepackage{xcolor}
\usepackage{hyperref}
\hypersetup{
    colorlinks=true,
    linkcolor=blue,
    citecolor=blue,
    urlcolor=blue
}

% Commandes personnalisées
\newcommand{\diff}{\mathrm{d}} % Pour les différentielles droites
\newcommand{\kB}{k_\mathrm{B}} % Constante de Boltzmann
\newcommand{\pV}{pV} % Produit pression-volume
\newcommand{\re}{\textrm}
\newcommand{\Cp}{\overline{C}_p} % definition de Cp molaire
\newcommand{\Cv}{\overline{C}_v} % definition de Cv molaire


\title{Formulaire de Thermodynamique}
\author{Raphaël Jamann}
\date{}

\begin{document}
\maketitle


\section{Definitions des grandeurs en Thermodynamique}

\begin{multicols}{2}

\subsection*{Système}

\textit{Un système est une portion de l'Univers que l'on isole du reste,
ce dernier constituant alors le milieu extérieur.}\\
Il y a trois types de système :
\begin{itemize}[label=\textbullet]
    \item \textbf{ouvert} s'il échange énergie et matière avec l'extérieur.
    \item \textbf{fermé} s'il n'échange que de l'énergie' avec l'extérieur.
    \item \textbf{isolé} s'il n'échange ni énergie, ni matière avec l'extérieur.
\end{itemize}

\subsection*{Variables d'état}
Un système est défini par des \textbf{variables d'état} qui peuvent être :
\begin{itemize}[label=\textbullet]
    \item \textbf{Intensives} : ne dépendent pas de la quantité de matière (ex: pression, température, etc\dots)
    \item \textbf{Extensives} : dépendent de la quantité de matière (ex: volume, énergie interne, entropie, etc\dots)
\end{itemize}

\noindent Voici quelques exemples de variables d'état :
\begin{itemize}[label=\textbullet]
    \item \textbf{Pression} : $P$ (en Pascal)
    \item \textbf{Volume} : $V$ (en m$^3$)
    \item \textbf{Température} : $T$ (en Kelvin)
    \item \textbf{Masse} : $m$ (en kg)
    \item \textbf{Nombre de moles} : $n$ (en mol)
    \item \textbf{Chaleur} : $Q$ (en J)
\end{itemize}

\subsection*{Énergie interne $\bm{U}$}
En joule, cette énergie est une fonction d'état (ne dépend pas du chemin suivie lors d'une transformation).
$$ \boxed{\diff U = \delta Q + \delta W} $$

\subsection*{Enthalpie : $\bm{H}$}
L'enthalpie est une fonction d'état extensive mesurée en joule.
$$ \boxed{H = U + pV} $$

Pour les processus effectués à pression constante, la variation d'enthalpie correspond à la chaleur absorbée (ou dégagée) pour rester à température constante : 
$$ \Delta H = Q_P $$


\subsection*{Entropie : $\bm{S}$}
L'entropie est une fonction d'état extensive mesurée en joule par Kelvin.\\
C'est est une grandeur physique qui caractérise le \textbf{degré de désorganisation d'un système}.
$$ \boxed{\Delta S = \int_{A}^{B} \dfrac{\delta Q_{\text{rév}}}{T}} $$


\end{multicols}


\section{Capacité Calorifique d'un système}

La capacité thermique est une grandeur extensive en $J.K^{-1}$.\\
\textit{C'est l'énergie à apporter pour élever la température d'un degré.}
\vspace{3mm}

\textbf{Pour les gaz}, comme ils sont compressibles, on distingue deux modes de chauffage :
\vspace{3pt}
\begin{itemize}[label=\textbullet]
    \item \textbf{Chauffage isobare :} on définis $\bm{C_p}$ (capacité calorifique du gaz à pression constante).
    \item \textbf{Chauffage isochore :} on définis $\bm{C_v}$ (capacité calorifique du gaz à volume constant).
\end{itemize}

\subsection*{Coefficient de Laplace d'un gaz : $\bm{\gamma}$}

\textit{Le coeficient de Laplace (ou l'indice adiabatique) d'un gaz est défini comme le rapport de
ses capacités thermiques à pression constante et à volume constant :} $\boxed{\gamma = \dfrac{C_p}{C_v}}$

Avec la \textbf{relation de Mayer} $C_P - C_V = R$ \ on peut en déduire que pour les gaz monoatomiques, $\gamma \approx 1,67$ et pour les gaz diatomiques, $\gamma \approx 1,4$.



\begin{multicols}{2}
[
    \section{Les Principes de la Thermodynamique}
    \textit{Un principe est une loi générale qui s'applique à tous les systèmes thermodynamiques. Il n'a pas été démontré, mais aucun résultats experimentaux ne l'a contredit.}
]

\subsection*{Le Premier Principe}
Conservation de l'énergie : $\boxed{\diff U = \delta Q + \delta W}$

\subsection*{Le Second Principe}
L'entropie d'un système isolé ne peut qu'augmenter : $\boxed{\diff S \geq 0}$

\subsection*{Le Troisième Principe}
L'entropie d'un cristal parfait à 0 K est nulle : $\boxed{S(0) = 0}$\\
L'entropie d'un système ne peut pas être inférieure à 0 : $\boxed{S \geq 0}$

\end{multicols}



\section{Gaz parfaits}

\subsection{Loi des Gazs Parfaits}
$$ \boxed{pV = nRT} $$



\section{Transformation d'un système entre deux états A et B}

Lors d'une transformation, le système peut échanger de l'énergie avec le mileu extérieur
sous forme de travail (noté $W$) ou de chaleur (noté $Q$).\\
On distingue deux types de transformations:
\begin{itemize}
    \item \textbf{réversible :} cas hypothétique d'une transformation infiniement lente. Donc le système est constament à l'équilibre avec l'extérieur.
    \item \textbf{irréverible :} Transformation "réelle" hors équilibre avec l'extérieur.
\end{itemize}



\subsection{Transformation Adiabatique}

\textbf{Pas d'échange de chaleur avec l'extérieur:} $\boxed{Q = 0}$\\ 
$\Longrightarrow \quad U = W$ \quad le travail se comporte comme une fonctiond d'état ($W_{\text{rév}} = W_{\text{irrév}}$)

\subsubsection{Si Réversible}

\begin{itemize}[label=\textbullet]
    \item \textbf{Relation de Poisson :} $PV^\gamma = \re{Cste}$
    \item \textbf{Relation de Laplace :} $T V^{\gamma - 1} = \re{Cste}$
    \item \textbf{Relation de Clapeyron :} $T P^{1 - \gamma} = \re{Cste}$
\end{itemize}

\subsection{Transformation Isotherme}


\subsection{Transformation Isobare}

\begin{itemize}[label=\textbullet]
    \item \textbf{Chaleur échangée :} $\delta Q = n \Cp \diff T$
    \item \textbf{Travail échangé :} $W = \int_{V_A}^{V_B}- P_{\text{ext}} \diff V = - P_{\text{ext}}(V_B-V_A)$
\end{itemize}

\subsection{Calcul du Travail d'un Transformation}

Dans le cas général: $ \boxed{\delta W = - P_{ext}\ \re d V} $



\end{document}