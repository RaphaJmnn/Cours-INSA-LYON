\documentclass[a4paper,12pt]{article}

% Encodage et langue
\usepackage[utf8]{inputenc}
\usepackage[T1]{fontenc}
\usepackage[french]{babel}

% Mathématiques
\usepackage{amsmath, amssymb, amsfonts}
\usepackage{siunitx} % Pour les unités
\sisetup{locale=FR} % Configuration pour le français
\usepackage{bm} % bold maths

% Graphiques et figures
\usepackage{graphicx}
\usepackage{enumitem} % pour les puces de listes
\usepackage{siunitx}

% Mise en page
\usepackage{geometry}
\geometry{a4paper, margin=1.5cm}
\usepackage{multicol}

% Couleurs et hyperliens
\usepackage{xcolor}
\usepackage{hyperref}
\hypersetup{
    colorlinks=true,
    linkcolor=blue,
    citecolor=blue,
    urlcolor=blue
}

% Commandes personnalisées
\newcommand{\diff}{\mathrm{d}} % Pour les différentielles droites
\newcommand{\kB}{k_\mathrm{B}} % Constante de Boltzmann
\newcommand{\pV}{pV} % Produit pression-volume
\newcommand{\re}{\textrm}
\newcommand{\Cp}{\overline{C}_p} % definition de Cp molaire
\newcommand{\Cv}{\overline{C}_v} % definition de Cv molaire


\title{Formulaire de Thermodynamique}
\author{Raphaël Jamann}
\date{}

\begin{document}
\maketitle


\section{Definitions des grandeurs en Thermodynamique}

    \begin{multicols}{2}

    \subsection*{Système}

    \textit{Un système est une portion de l'Univers que l'on isole du reste,
    ce dernier constituant alors le milieu extérieur.}\\
    Il y a trois types de système :
    \begin{itemize}[label=\textbullet]
        \item \textbf{ouvert} s'il échange énergie et matière avec l'extérieur.
        \item \textbf{fermé} s'il n'échange que de l'énergie' avec l'extérieur.
        \item \textbf{isolé} s'il n'échange ni énergie, ni matière avec l'extérieur.
    \end{itemize}

    \subsection*{Variables d'état}
    Un système est défini par des \textbf{variables d'état} qui peuvent être :
    \begin{itemize}[label=\textbullet]
        \item \textbf{Intensives} : ne dépendent pas de la quantité de matière (ex: pression, température, etc\dots)
        \item \textbf{Extensives} : dépendent de la quantité de matière (ex: volume, énergie interne, entropie, etc\dots)
    \end{itemize}

    \noindent Voici quelques exemples de variables d'état :
    \begin{itemize}[label=\textbullet]
        \item \textbf{Pression} : $P$ (en Pascal)
        \item \textbf{Volume} : $V$ (en m$^3$)
        \item \textbf{Température} : $T$ (en Kelvin)
        \item \textbf{Masse} : $m$ (en kg)
        \item \textbf{Nombre de moles} : $n$ (en mol)
        \item \textbf{Chaleur} : $Q$ (en J)
    \end{itemize}

    \subsection*{Énergie interne $\bm{U}$}
        En joule, cette énergie est une fonction d'état (ne dépend pas du chemin suivie lors d'une transformation).
        $$ \boxed{ \diff U = C_v \, \diff T + (l - P) \, \diff V } $$

    \subsection*{Chaleur $\bm{Q}$}
        Energie calorifique échangé au cours d'une transformation.
            $$ \boxed{ Q_{\text{rév}} = C_v \, \diff T + l \, \diff V } $$
            $$ \boxed{ Q_{\text{rév}} = C_p \, \diff T + h \, \diff P } $$

    \subsection*{Travail $\bm{W}$}
        Energie non calorifique échangé au cours d'une transformation :
            $$ \boxed{\delta W = - P_{ext}\ \re d V} $$

    \subsection*{Enthalpie $\bm{H}$}
        L'enthalpie est une fonction d'état extensive mesurée en joule.
            $$ \boxed{H = U + P V} $$
            $$ \boxed{ \diff H = C_p \, \diff T + (h + V) \, \diff P } $$

    \subsection*{Entropie : $\bm{S}$}
        L'entropie est une fonction d'état extensive mesurée en joule par Kelvin.\\
        C'est est une grandeur physique qui caractérise le \textbf{degré de désorganisation d'un système}.
        $$ \boxed{\diff S_\sigma = \frac{\delta Q_{\sigma,\text{rév}}}{T_\sigma}} $$

    \end{multicols}

\clearpage

\section{Capacité Calorifique d'un système}

    La capacité thermique est une grandeur extensive en $J.K^{-1}$.\\
    \textit{C'est l'énergie à apporter pour élever la température d'un degré.}
    \vspace{3mm}

    \textbf{Pour les gaz}, comme ils sont compressibles, on distingue deux modes de chauffage :
    \vspace{3pt}
    \begin{itemize}[label=\textbullet]
        \item \textbf{Chauffage isobare :} on définis $\bm{C_p}$ (capacité calorifique du gaz à pression constante).
        \item \textbf{Chauffage isochore :} on définis $\bm{C_v}$ (capacité calorifique du gaz à volume constant).
    \end{itemize}

    \subsection*{Coefficient de Laplace d'un gaz : $\bm{\gamma}$}

    \textit{Le coeficient de Laplace (ou l'indice adiabatique) d'un gaz est défini comme le rapport de
    ses capacités thermiques à pression constante et à volume constant :} $\boxed{\gamma = \dfrac{C_p}{C_v}}$

    Avec la \textbf{relation de Mayer} $C_P - C_V = R$ \ on peut en déduire que pour les gaz monoatomiques, $\gamma \approx 1,67$ et pour les gaz diatomiques, $\gamma \approx 1,4$.




\section{Les Principes de la Thermodynamique}

    \textit{Un principe est une loi générale qui s'applique à tous les systèmes thermodynamiques. Il n'a pas été démontré, mais aucun résultat experimental ne l'a contredit.}

    \begin{multicols}{2}

        \subsection*{Le Premier Principe}
        Conservation de l'énergie : $\boxed{\diff U = \delta Q + \delta W}$

        \subsection*{Le Second Principe}
        L'entropie d'un système isolé ne peut qu'augmenter : $\boxed{\diff S \geq 0}$ (créable et indestructible).

        \subsection*{Le Troisième Principe}
        L'entropie d'un cristal parfait à 0 K est nulle : $\boxed{S(0) = 0}$\\
        L'entropie d'un système ne peut pas être inférieure à 0 : $\boxed{S \geq 0}$

    \end{multicols}



\section{Gaz parfaits}

    \begin{multicols}{2}
        
        \subsection*{Loi des Gazs Parfaits}
            $$ \boxed{PV = nRT} $$
            Avec $P$ en Pascals, $V$ en $m^3$, $n$ en mol, $ R=\qty{8.314}{J mol^{-1} K^{-1}} $ et $T$ en Kelvin. 
        
        \subsection*{Energie interne}
            $$ \boxed{ \diff U_{GP} = C_v \, \diff T } $$
            Avec $\Delta U$ en Joules.

        \subsection*{Chaleur $\bm{Q}$}
            Pour un gaz parfait, comme la chaleur latente d'expension $l=P$ et que la chaleur latente de compression $h=-V$ on a :
                $$ \boxed{ Q_{\text{GP,rév}} = C_v \, \diff T + P \, \diff V } $$
                $$ \boxed{ Q_{\text{GP,rév}} = C_p \, \diff T - V \, \diff P } $$

        \subsection*{Enthalpie}
            $$ \boxed{ \diff H_{GP} = C_p \, \diff P } $$

    \end{multicols}

\clearpage

\section{Transformation d'un gaz parfait entre deux états A et B}

    Lors d'une transformation, le système peut échanger de l'énergie avec le milieu extérieur
    sous forme de travail (noté $W$) ou de chaleur (noté $Q$).\\
    On distingue deux types de transformations:
    \begin{itemize}
        \item \textbf{réversible :} cas hypothétique d'une transformation infiniment lente. Donc le système est constament à l'équilibre avec l'extérieur.
        \item \textbf{irréversible :} Transformation "réelle" hors équilibre avec l'extérieur.
    \end{itemize}


    \subsection{Transformation Adiabatique}

        \textbf{Pas d'échange de chaleur avec l'extérieur:} $\boxed{Q = 0}$\\ 
        $ \Longrightarrow \quad \Delta U = W = n\Cv\Delta T $ \quad le travail se comporte comme une fonction d'état ($W_{\text{rév}} = W_{\text{irrév}}$).

        \subsubsection{Réversible}

            \begin{itemize}[label=\textbullet]
                \item \textbf{Relation de Poisson :} $PV^\gamma = \re{cste}$
                \item \textbf{Relation de Laplace :} $T V^{\gamma - 1} = \re{cste}$
                \item \textbf{Relation de Clapeyron :} $T^\gamma P^{1 - \gamma} = \re{cste}$
            \end{itemize}

    \subsection{Transformation Isotherme}

        Transformation à température constante ($T_A=T_B$).
        
        Comme $ \Delta U_{GP} = 0 \quad (\Delta T = 0) $ on a $W_{AB} = -Q_{AB}$.

        \subsubsection{Réversible}
            $$ W_{AB} = \int_{V_A}^{V_B} -P_{ext} \, \diff V = \int_{V_A}^{V_B} -P_\sigma \, \diff V = \int_{V_A}^{V_B} -\frac{nRT}{V} \, \diff V = -nRT \int_{V_A}^{V_B} \frac{\diff V}{V} = -nRT \ln{\left(\frac{V_B}{V_A}\right)} $$
            
            $$ \Delta S_\sigma = \int_A^B \frac{\delta Q_{\sigma,\text{rév}}}{T_\sigma} = \int_{V_A}^{V_B} \frac{P_\sigma \, \diff V}{T_\sigma} = \int_A^B \frac{-\delta W_{\text{rév}}}{T_\sigma} = n R \ln{\left(\frac{V_B}{V_A}\right)} $$

            $$ \Delta S_{\text{ext}} = \int_A^B \frac{\delta Q_{\text{ext,rév}}}{T_{\text{ext}}} = \int_A^B \frac{-\delta Q_{\sigma,\text{rév}}}{T_{\text{ext}}} = - n R \ln{\left(\frac{V_B}{V_A}\right)} $$

            On retrouve bien que pour une transformation réversible: $\Delta U_u = 0$. (On aurait pu directement dire que $\Delta U_{\text{ext}} = -\Delta U_\sigma$)

        \subsubsection{Irréversible}
            $$ W_{AB'} = \int_{V_A}^{V_B} -P_{ext} \, \diff V = -P_{ext} \int_{V_A}^{V_B} \diff V = -P_{ext} (V_B - V_A) = P_{ext} (V_A- V_B) $$

            $$ \Delta S_\sigma' = \int_A^B \frac{\delta Q_{\sigma,\text{rév}}}{T_\sigma} = \int_{V_A}^{V_B} \frac{P \, \diff V}{T_\sigma} = \int_A^B \frac{-\delta W_{AB}}{T_\sigma} = n R \ln{\left(\frac{V_B}{V_A}\right)} $$

            On retrouve bien que $\Delta S_\sigma' = \Delta S_\sigma$ puisque l'entropie est une fonction d'état.
            
            $$ \Delta S_{\text{ext}}' = \int_A^B \frac{\delta Q_{\text{ext,rév}}}{T_{\text{ext}}} = \int_A^B \frac{-\delta Q_{\sigma,\text{irrév}}}{T_{\text{ext}}} = \int_A^B \frac{\delta W_{\sigma,\text{irrév}}}{T_{\text{ext}}} = -\frac{P_B (V_B - V_A)}{T_{\text{ext}}} $$

            $$ \Delta U_u' = \frac{-W_{\text{rév}}}{T_{\text{ext}}} + \frac{W_{\text{irrév}}}{T_{\text{ext}}} = \frac{W_{\text{irrév}} - W_{\text{rév}}}{T_{\text{ext}}} > 0 $$

    \subsection{Transformation Isobare}
        
        Transformation à pression constante ($P_A=P_B$).

        \subsubsection{Réversible}
            $$ Q_{AB} = n \Cp \, \Delta T - V \, \Delta P = n \Cp \, \Delta T $$
        
        \subsubsection{Irréversible}
            $$ W_{AB'} = \int_{V_A}^{V_B} -P_{\text{ext}} \diff V = -P_{\text{ext}}(V_B-V_A) = P_{\text{ext}}(V_A-V_B) $$

    \subsection{Transformation Isochore}

        Transformation à volume constant ($V_A=V_B$).

        Dans tout les cas : $ W_{AB} = 0 $ car $ \diff V = 0 $. Donc on a $ \Delta U_{GP} = Q_{AB} $.
        
        Dans tout les cas : $Q_{AB} = \Delta H_{GP} = C_p \, \Delta T $
                
        \subsubsection{Réversible}

             $$ Q_{AB} = n \Cp \, \Delta T - V \, \Delta P = n \Cp \, \Delta T $$

\end{document}